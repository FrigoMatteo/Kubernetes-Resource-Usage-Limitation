\section{Introduction}
The objective of this lab was created by the Cybersecurity course of Politecnico di Torino, which requires the complete creation of a laboratory activity based on the concepts of “domains of protection” or “multi-tenancy”, starting from the ResourceQuotas feature of Kubernetes and searching for other technologies for resource usage limitation. \\
In today’s world, it's essential to have a system that lets us logically divide and isolate different teams, groups of users, or user work on the same machine, reducing the cost of hardware and management. In fact, this lab aims at understanding and practicing the instruments enabling multi-tenancy in the Kubernetes environment. \\
The focus of the project is resource usage limitation, where we don't focus only on resource differentiation, but also on the constrainment and isolation of multiple teams inside our Kubernetes cluster. This will require instruments that let us logically divide the environments of different teams by using ResourceQuota, LimitRange, Access Control, and Network Policy. We will understand those tools, providing also some use case examples. All the tools we will see can be combined together inside a single cluster, fundamental to providing an advanced multi-tenancy environment for multiple customers.